Este es el resumen de la aplicación. La siguiente aplicación esta conformada por dos aplicaciones que conforma la conexion entre un cliente y un servidor que tiene el baash desarrollado en S\+OI. Lenguaje\+: C.\section*{cliente y servidor }

\begin{DoxyAuthor}{Author}
Sepulveda Federico (\href{mailto:federico.sepulveda@alumnos.unc.edu.ar}{\tt federico.\+sepulveda@alumnos.\+unc.\+edu.\+ar}) 
\end{DoxyAuthor}
\begin{DoxyDate}{Date}
Abril, 2018
\end{DoxyDate}
\begin{DoxyParagraph}{I\+N\+T\+R\+O\+D\+U\+C\+C\+IÓN\+:}
Una de las aplicaciones (el Cliente), se utiliza para conectarse a un Servidor, que es la creado por la otra aplicación.~\newline
Se comunican via socket y se pueden ejecutar comandos como si tratara de una terminal de G\+NU linux. Ademas se le agrego la funcion \char`\"{}descarga nombre\+\_\+archivo\char`\"{} que perminte descargar desde el servirdor el archivo deseado.
\end{DoxyParagraph}
\begin{DoxyParagraph}{I\+N\+S\+T\+A\+L\+A\+C\+IÓN\+:}
(A\+V\+I\+SO\+: tanto la aplicacion cliente como servidor se encuentran en dierectorios distintintos)~\newline
A continuación se detallan algunos comandos útiles para el uso de la aplicación\+:~\newline
\begin{DoxyVerb}"ip addr"                               --> Brinda informacion acerca de las redes del equipo. Es util para averiguar la 
                                            direccion IPv4 del servidor.

En el directorio /tp1/tp1_servidor_baash ejecutar:
"make"                                  --> Compila el proyecto y genera ejecutable del server.
"./tp1_servidor_baash 6020"             --> Corre el server y comienza a escuchar por el puerto 6020.

En el directorio /tp1/tp1_cliente_baash ejecutar:
"make"                                  --> Compila el proyecto y genera ejecutable del cliente.
"./tp1_cliente_baash"                   --> Corre el cliente.
\end{DoxyVerb}

\end{DoxyParagraph}
\begin{DoxyParagraph}{U\+T\+I\+L\+I\+Z\+A\+C\+IÓN\+:}
(A\+V\+I\+SO\+: todos los comandos se ingresan desde el cliente)~\newline
A continuacion se listan algunos comandos que reconoce la aplicacion\+:~\newline
\begin{DoxyVerb}"descarga nombre_archivo"               --> Descarga en el cliente el archivo solicitado al servidor.
"ls"                                    --> Muestra los archivos del directorio actual.
"ps"                                    --> Muestra los procesos que estan corriendo.
"date"                                  --> Muestra la fecha que tiene el servidor.
\end{DoxyVerb}

\end{DoxyParagraph}
\begin{DoxyParagraph}{E\+J\+E\+M\+P\+LO\+:}
(En el Cliente)~\newline
A continuación se detalla una secuencia de comandos típica a modo de ejemplo\+:~\newline
\begin{DoxyVerb}"connect usuario@numero_ip:6020"    --> Se conecta al servidor con IP=numero_ip. (averiguar con ip addr)
"pass"                              --> Clave correspondiente al usuario 'usuario'. El servidor solo permite 4 (cuatro) intentos 
                                    para ingresar la contraseña, una vez superado este numero se cierra la conexion.            
"ls"                                --> El servidor respondera con un lista con los nombres de los archivos dentro del directorio actual\end{DoxyVerb}
 
\end{DoxyParagraph}
