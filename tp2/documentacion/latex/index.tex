Esta aplicacion en un procesador dopple, que lee los datos que son muestra de un A\+DC de un archivo binario pulso.\+iq y devuelve un resultado en un archivo binaro que contiene la autocorrelacion de los pulsos recibidos. Lenguaje\+: C.===================

\begin{DoxyAuthor}{Author}
Sepulveda Federico (\href{mailto:federico.sepulveda@alumnos.unc.edu.ar}{\tt federico.\+sepulveda@alumnos.\+unc.\+edu.\+ar}) 
\end{DoxyAuthor}
\begin{DoxyDate}{Date}
Mayo, 2018
\end{DoxyDate}
\begin{DoxyParagraph}{I\+N\+T\+R\+O\+D\+U\+C\+C\+IÓN\+:}
Esta aplicacion esta desarrollada de dos maneras, una sin explotar el paralelismo y la otra explotando el paralelismo utilizando la biblioteca Open\+M\+PI
\end{DoxyParagraph}
\begin{DoxyParagraph}{I\+N\+S\+T\+A\+L\+A\+C\+IÓN y U\+T\+I\+L\+I\+Z\+A\+C\+IÓN\+:}
Para la aplicaion procedural solo basta con ejecutar la aplicacion y esta genera un archivo binario llamado resultados con los datos~\newline
A continuación se detallan la ejecucion\+:~\newline
\begin{DoxyVerb}En el directorio /tp1/tp1_servidor_baash ejecutar:
"make"                                  --> Compila el proyecto y genera ejecutable.
"./tp2_procedural"              --> Corre la aplicacion.
\end{DoxyVerb}

\end{DoxyParagraph}
Para la aplicaion paralela se debe ejecutar pasando como parametro la cantidad de hilos~\newline
y esta genera un archivo binario llamado resultados con los datos~\newline
\begin{DoxyVerb}En el directorio /tp1/tp1_cliente_baash ejecutar:
"make"                                  --> Compila el proyecto y genera ejecutable.
"./tp2_procedural <Numero de Hilos>"                    --> Corre la aplicacion con la cantidad de hilos pasada como parametro.\end{DoxyVerb}
 